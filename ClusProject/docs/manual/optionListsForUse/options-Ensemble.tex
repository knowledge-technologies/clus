\section{Ensemble}


\begin{itemize}
    \item \optionNameStyle{Iterations}:
           \begin{itemize}
                \item \optionPossibleValues{}: positive integer $n$, $n \geq 2$, or a list of such integers
                \item \optionDefaultValue{}: \optionDefaultValueStyle{0}
                \item \optionDescrption{}: defines the number of base-level models (trees) in the ensemble. If this
                setting is given as a list of numbers $[n_1, n_2, \dots]$, such that $n_1 < n_2 < \cdots$, an ensemble $\mathcal{E}_i$ for each element of the list is computed in an efficient way, so that $\mathcal{E}_1$ consist of $n_1$ trees and is contained in $\mathcal{E}_2$ etc.
           \end{itemize}
    \item \optionNameStyle{EnsembleMethod}:
           \begin{itemize}
                \item \optionPossibleValues{}: an element from \optionPossibleValuesList{Bagging,RForest,RSubspaces,BagSubspace,Boosting,RFeatSelection,Pert,ExtraTrees}
                \item \optionDefaultValue{}: \optionDefaultValueStyle{Bagging}
                \item \optionDescrption{}: defines the ensemble method.
%                \begin{itemize}
%                    \item 
%                    \item \formatOneElement{RSubspaces}: rnd descriptive 
%                    \item \formatOneElement{BagSubspace}
%                \end{itemize}
           \end{itemize}
    \item \optionNameStyle{VotingType}:
           \begin{itemize}
                \item \optionPossibleValues{}: an element of \optionPossibleValuesList{Majority, ProbabilityDistribution}
                \item \optionDefaultValue{}: \optionDefaultValueStyle{ProbabilityDistribution}
                \item \optionDescrption{}:
                	\begin{itemize}
                		\item \formatOneElement{Majority}: each base-level model casts one vote, for regression this is equal to averaging.
                		\item \formatOneElement{ProbabilityDistribution}: each base-level model casts probability distributions for each target attribute, does not work for regression.
                	\end{itemize}
           \end{itemize}
    \item \optionNameStyle{SelectRandomSubspaces}:
           \begin{itemize}
                \item \optionPossibleValues{}: an integer $k$ from the interval $[1, \mathrm{number\ of\ attributes}]$ or a real number $q$ from the interval $[0, 1]$, or an element
                of \optionPossibleValuesList{LOG, SQRT}%, RANDOM}

                \item \optionDefaultValue{}: \optionDefaultValueStyle{0}
                \item \optionDescrption{}:
                 \begin{itemize}
                    \item \formatOneElement{LOG} or \formatOneElement{0}: the feature subset size is set to $\lceil{\log_2(\mathrm{number\ of\ descriptive\ attributes})}\rceil$ as recommended by Breiman \cite{Breiman2001}. This is the default value.
                    \item \formatOneElement{SQRT}: the feature subset size is set to $\lceil{\sqrt{\mathrm{number\ of\ descriptive\ attributes}}}\rceil$.
                    %\item \formatOneElement{RANDOM}: works only if \formatOneElement{EnsembleMethod} is set to \formatOneElement{RSubspaces}
                    \item $k$: an integer number that specifies the feature subset size.
                    \item $q$: the fraction of the descriptive attributes used as feature subset, i.e., the feature subset value is set to $[{q(\mathrm{number\ of\ descriptive\ attributes})}]$ .
                \end{itemize}
           \end{itemize}
    \doNotShowThis{
    \item \optionNameStyle{SelectRandomTargetSubspaces}:
           \begin{itemize}
                \item \optionPossibleValues{}: ???
                \item \optionDefaultValue{}: \optionDefaultValueStyle{SQRT}
                \item \optionDescrption{}: ???
           \end{itemize}
    \item \optionNameStyle{RandomOutputSelection}:
           \begin{itemize}
                \item \optionPossibleValues{}: ???
                \item \optionDefaultValue{}: \optionDefaultValueStyle{None}
                \item \optionDescrption{}: ???
           \end{itemize}
    }
    \item \optionNameStyle{PrintAllModels}:
           \begin{itemize}
                \item \optionPossibleValues{}: an element of  \optionPossibleValuesList{Yes, No}
                \item \optionDefaultValue{}: \optionDefaultValueStyle{No}
                \item \optionDescrption{}: specifies whether the base models are included in the \texttt{.out} file.
           \end{itemize}
    \item \optionNameStyle{PrintAllModelFiles}:
           \begin{itemize}
                \item \optionPossibleValues{}: an element of \optionPossibleValuesList{Yes, No}
                \item \optionDefaultValue{}: \optionDefaultValueStyle{No}
                \item \optionDescrption{}: If \formatOneElement{Yes}, \clus\ will save all base-level models of an ensemble in the model file. The default setting prevents creating very large model files.
           \end{itemize}
    \item \optionNameStyle{PrintAllModelInfo}:
           \begin{itemize}
                \item \optionPossibleValues{}: an element of \optionPossibleValuesList{Yes, No}
                \item \optionDefaultValue{}: \optionDefaultValueStyle{No}
                \item \optionDescrption{}: If \formatOneElement{Yes}, \clus\ will include additional model information in the \texttt{.out} file.
           \end{itemize}
    \item \optionNameStyle{PrintPaths}:
           \begin{itemize}
                \item \optionPossibleValues{}: an element of \optionPossibleValuesList{Yes, No}
                \item \optionDefaultValue{}: \optionDefaultValueStyle{No}
                \item \optionDescrption{}:If \formatOneElement{Yes}, for each tree, the path that is followed by each instance in that tree will be printed to a file named {\tt tree\_i.path}.  The default setting is \texttt{No}. Currently, the setting is only implemented for \formatOneElement{Bagging} and \formatOneElement{RForest}. This setting was used in random forest based feature induction, see \cite{Vens2011} for details
                %, and the following directory for an example:  
                %	\begin{flushleft}
                %		\verb^$CLUS_DIR/data/rfbfi/.^
                %	\end{flushleft}
           \end{itemize}
    \item \optionNameStyle{Optimize}:
           \begin{itemize}
                \item \optionPossibleValues{}: an element of \optionPossibleValuesList{Yes, No}
                \item \optionDefaultValue{}: \optionDefaultValueStyle{No}
                \item \optionDescrption{}: If set to \formatOneElement{Yes}, \clus\ will optimize memory usage during learning. In that case, the tree induction/predictive error computation procedure is slightly modified.
                For example, the predictions of the ensemble are updated every time a new tree is grown.
                After that, the tree is removed from the memory. Due to floating point arithmetic, this could result in slightly different predictions.
           \end{itemize}
    \item \optionNameStyle{OOBestimate}:
           \begin{itemize}
                \item \optionPossibleValues{}: an element of \optionPossibleValuesList{Yes, No}
                \item \optionDefaultValue{}: \optionDefaultValueStyle{No}
                \item \optionDescrption{}:  If \formatOneElement{Yes}, out-of-bag estimate of the performance of the ensemble will be done.
           \end{itemize}
    \item \optionNameStyle{FeatureRanking}:
           \begin{itemize}
                \item \optionPossibleValues{}:  an element of \optionPossibleValuesList{None, RForest, GENIE3, SYMBOLIC}
                \item \optionDefaultValue{}: \optionDefaultValueStyle{None}
                \item \optionDescrption{}: If set to non-default value, the specified ranking will be computed in addition to the ensemble predictive model.
                    \begin{itemize}
                        \item \formatOneElement{RForest}: random forest ranking as described in \cite{petkovic2017-ensemble-fr}, does not work with extra trees (no out-of-bag examples)
                        \item \formatOneElement{GENIE3}: Genie3 ranking as given in \cite{petkovic2017-ensemble-fr}
                        \item \formatOneElement{SYMBOLIC}: Symbolic ranking, as described in \cite{petkovic2017-ensemble-fr}
                    \end{itemize}
           \end{itemize}
    \item \optionNameStyle{FeatureRankingPerTarget}:
           \begin{itemize}
                \item \optionPossibleValues{}:an element of  \optionPossibleValuesList{Yes, No}
                \item \optionDefaultValue{}: \optionDefaultValueStyle{No}
                \item \optionDescrption{}: If the \formatOneElement{FeatureRanking}, the \textit{number of targets} additional feature rankings are computed, each one only with respect to one target.
                This option is ignored when the additional rankings are the same as the original one, e.g., when there is only one target or \formatOneElement{FeatureRanking} is set to \formatOneElement{SYMBOLIC}.
           \end{itemize}
    \item \optionNameStyle{SymbolicWeight}:
           \begin{itemize}
                \item \optionPossibleValues{}: a real number $w$ from the interval $(0, 1]$.
                \item \optionDefaultValue{}: \optionDefaultValueStyle{1.0}
                \item \optionDescrption{}: every appearance of an attribute at the depth $d$ in a tree increases the importance score of \formatOneElement{SYMBOLIC} by $w^d$. If \optionNameStyle{FeatureRanking} is not
                set to \formatOneElement{SYMBOLIC}, this option has no influence.
           \end{itemize}
    \item \optionNameStyle{SortRankingByRelevance}:
           \begin{itemize}
                \item \optionPossibleValues{}: an element of \optionPossibleValuesList{Yes, No}
                \item \optionDefaultValue{}: \optionDefaultValueStyle{Yes}
                \item \optionDescrption{}: If set to \formatOneElement{Yes}, the attributes in the \texttt{.fimp} file are sorted decreasingly by relevance. Otherwise, they are in the same as in \texttt{.arff} file.
           \end{itemize}
    \item \optionNameStyle{WriteEnsemblePredictions}:
           \begin{itemize}
                \item \optionPossibleValues{}: an element of \optionPossibleValuesList{Yes, No}
                \item \optionDefaultValue{}: \optionDefaultValueStyle{No}
                \item \optionDescrption{}: If set to \formatOneElement{Yes}, the predictions of the ensemble for the training set are written to \texttt{{\em filename}.ens.train.preds}.
                If the option \formatOneElement{TestSet} is also specified,  the predictions of the ensemble for the testing set are written to \texttt{{\em filename}.ens.test.preds}.
           \end{itemize}
    \doNotShowThis{
    \item \optionNameStyle{EnsembleRandomDepth}:
           \begin{itemize}
                \item \optionPossibleValues{}: ???
                \item \optionDefaultValue{}: \optionDefaultValueStyle{No}
                \item \optionDescrption{}: ???
           \end{itemize}
    \item \optionNameStyle{BagSelection}:
           \begin{itemize}
                \item \optionPossibleValues{}: ???
                \item \optionDefaultValue{}: \optionDefaultValueStyle{-1}
                \item \optionDescrption{}: ???
           \end{itemize}
    }
    \item \optionNameStyle{BagSize}:
           \begin{itemize}
                \item \optionPossibleValues{}: an integer from the interval $[0, \mathrm{number\ of\ instances}]$
                \item \optionDefaultValue{}: \optionDefaultValueStyle{0}
                \item \optionDescrption{}: Specifies the size of a bag. If set to $0$, this will be converted to the \emph{number of instances}.
           \end{itemize}
    \item \optionNameStyle{NumberOfThreads}:
           \begin{itemize}
                \item \optionPossibleValues{}: An integer, greater or equal to $1$
                \item \optionDefaultValue{}: \optionDefaultValueStyle{1}
                \item \optionDescrption{}: Specifies the number of threads that run in parallel when inducing the ensemble from \optionPossibleValuesList{Bagging, RForest, ExtraTrees},
                or computing \formatOneElement{Relief} feature ranking.
           \end{itemize}
\end{itemize}




%\begin{itemize}
%
%	\item \texttt{EnsembleMethod = $o$} : $o$ is an element of \{\texttt{Bagging, RForest, RSubspaces, BagSubspaces}\} defines the ensemble method.
%	\begin{itemize}
%		\item \texttt{Bagging}: Bagging \cite{Breiman1996}.
%		\item \texttt{RForest}: Random forest \cite{Breiman2001}.
%		\item \texttt{RSubspaces}: Random Subspaces \cite{Ho1998}.
%		\item \texttt{BagSubspaces}: Bagging of subspaces \cite{PanovDzeroski2007}.
%		\item \texttt{Extra-Trees}: Extra trees
%	\end{itemize}
%	\item \texttt{BagSize = $n$}: When using a bagging scheme in large datasets, it might be useful to control the size of the individual bags. For $n > 0$, bags will have size $n$ rather than size $N$ (the size of the dataset).
%	\item \texttt{VotingType = $o$} : $o$ is an element of \{\texttt{Majority, ProbabilityDistribution}\} selects the voting scheme for combining predictions of base-level models.
%	\begin{itemize}
%		\item \texttt{Majority}: each base-level model casts one vote, for regression this is equal to averageing.
%		\item \texttt{ProbabilityDistribution}: each base-level model casts probability distributions for each target attribute, does not work for regression.
%	\end{itemize}
%	The default value is \texttt{Majority}, Bauer and Kohavi \cite{BauerKohavi1999} recommend \texttt{ProbabilityDistribution}.
%	\item \texttt{SelectRandomSubspaces = $o$} : $o$ is an element of \{\texttt{log, sqrt, 0, int, double}\} and defines size of feature subset for random forests, random subspaces and bagging of subspaces. 
%	\begin{itemize}
%		\item \texttt{log} or \texttt{0}: the feature subset size is set to $\lceil{\log_2(\mathrm{number\ of\ descriptive\ attributes})}\rceil$ as recommended by Breiman \cite{Breiman2001}. This is the default value.
%		\item \texttt{sqrt}: the feature subset size is set to $\lceil{\sqrt{\mathrm{number\ of\ descriptive\ attributes}}}\rceil$.
%		\item \texttt{int}: an integer number that specifies the feature subset size (the maximum allowed is the $\mathrm{number\ of\ descriptive\ attributes}$).
%		\item \texttt{double}: an double value $dval$ from the range (0,1) that specifies the fraction of the descriptive attributes used as feature sub set (the maximum allowed is 1.0), i.e., the feature subset value is set to $[{dval \cdot (\mathrm{number\ of\ descriptive\ attributes})}]$ .
%	\end{itemize}
%	\item {\tt PrintAllModels = $y$} : If \texttt{Yes}, \clus\ will print all base-level models of an ensemble in the output file. The default setting is \texttt{No}.
%	\item {\tt PrintAllModelFiles = $y$}: If \texttt{Yes}, \clus\ will save all base-level models of an ensemble in the model file. The default setting is \texttt{No}, which prevents from creating very large model files.
%	\item {\tt Optimize = $y$} : If \texttt{Yes}, \clus\ will optimize memory usage during learning. The default setting is \texttt{No}.
%	\item {\tt OOBestimate = $y$} : If \texttt{Yes}, out-of-bag estimate of the performance of the ensemble will be done. The default setting is \texttt{No}.
%	\item {\tt FeatureRanking = $y$} : If \texttt{Yes}, feature ranking via random forests will be performed. The default setting is \texttt{No}.
%	\item {\tt EnsembleRandomDepth = $y$} : If \texttt{Yes}, different random depth for each base-level model is selected. Used, e.g., in rule ensembles. The \texttt{MaxDepth} setting from \texttt{[Tree]} section is used as average. The default setting is \texttt{No}.
%	\item {\tt PrintPaths = $y$} : If \texttt{Yes}, for each tree, the path that is followed by each instance in that tree will be printed to a file named {\tt tree\_i.path}.  The default setting is \texttt{No}. Currently, the setting is only implemented for bagging and random forests. This setting was used in random forest based feature induction, see \cite{Vens2011} for details, and the following directory for an example:  
%	\begin{flushleft}
%		\verb^$CLUS_DIR/data/rfbfi/.^
%	\end{flushleft}
%
%
%%BagSelection = -1
%% 'Quasi' parallel implementation for ensembles
%% its value is an integer ranging from -1 to the number of trees in the ensemble
%% if set to -1, then the parallel implementation is discarded
%% if set to 0, then CLUS combines the predictions from the different bags
%% if set to i, i in 1..EnsembleSize, then CLUS learns the model for the i-th bag
%% if set as interval [val1-val2], this means to learn all bags between val1 and val2, including val1 and val2
%
%
%\end{itemize}
