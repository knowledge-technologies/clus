\section{Relief}
Note that one must ese \texttt{-relief} option to run the Relief algorithm.

\begin{itemize}
    \item \optionNameStyle{Neighbours}:
           \begin{itemize}
                \item \optionPossibleValues{}: integer between $1$ and the number of training instances, or a list of such integers
                \item \optionDefaultValue{}: \optionDefaultValueStyle{10}
                \item \optionDescrption{}: This is the number of neighbours in the Relief algorithms. A warning is given
                                           when this falls out of bounds. In that case, we compute a ranking with the default value for this option.
                                           The same goes for every element of the list. In that case, the computational complexity of the algorithm
                                           approximately equals the complexity of Relief with the maximal number of neighbours.
           \end{itemize}
    \item \optionNameStyle{Iterations}:
           \begin{itemize}
                \item \optionPossibleValues{}: integer between $1$ and the number of instances
                \item \optionDefaultValue{}: \optionDefaultValueStyle{0.0}
                \item \optionDescrption{}: This is the number of iterations in the Relief algorithms. If it is given as a proportion
                                           of all training instances, we convert it to the absolute value by rounding to the nearest integer.
                                           If only the value $1$ (or $1.0$) is given, we assume this means $100\%$ in both cases. A warning is given
                                           when this falls out of bounds. In that case, we compute a ranking with the default value for this option.
                                           The same goes for every element of the list. In that case, the computational complexity of the algorithm
                                           approximately equals the complexity of Relief with the maximal number of iterations.
           \end{itemize}
    \item \optionNameStyle{WeightNeighbours}:
           \begin{itemize}
                \item \optionPossibleValues{}: {\tt Yes} or {\tt No}.
                \item \optionDefaultValue{}: \optionDefaultValueStyle{No}
                \item \optionDescrption{}: Boolean that determines whether we weight the influence of the neighbours. If set to {\tt No},
                                           the value specified under the option \optionNameStyle{WeightingSigma} has no influence.
           \end{itemize}
    \item \optionNameStyle{WeightingSigma}:
           \begin{itemize}
                \item \optionPossibleValues{}: any non-negative real number
                \item \optionDefaultValue{}: \optionDefaultValueStyle{0.5}
                \item \optionDescrption{}: determines how quickly the influence of the neighbours decreases with their position $i$ (nearest, 2nd nearest etc.).
                                           The weights of the neighbours are proportional to $\exp(-(\sigma i)^2)$, so the value $0.0$ corresponds to
                                           the case when  \optionNameStyle{WeightNeighbours} is set to {\tt No}.
           \end{itemize}
\end{itemize}
