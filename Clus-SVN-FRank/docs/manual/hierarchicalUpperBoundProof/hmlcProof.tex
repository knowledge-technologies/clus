\documentclass[12pt,a4paper]{article}
\usepackage[utf8]{inputenc}
\usepackage{amsmath,amsthm,amsfonts,amssymb}
\usepackage{algorithm,algorithmic}
%opening
\title{Proof of correctness of the algorithm for finding the upper bound for label hierarchies}
\author{Matej Petković}

\begin{document}

\maketitle

We have a hierarchical attribute with labels $\mathcal{L} = \{\ell_1, \ell_2, \dots, \ell_n\}$, and some examples
in a dataset $\mathcal{D}$. Distance $d(x,y)$ between two $0/1$-vectors that denote presence/absence of a labels,
is given as
$$
d(x, y)^2 = \sum_i w_i (x_i - y_i)^2\text{,}
$$
where we assume that the weights $w_i$ are positive and increase with depth. Our goal is to compute

\begin{equation}
\label{eqn:goal}
\max_{x, y\in\mathcal{D}} d(x, y)
\end{equation}
Two notes: (i) do not mind the abuse of notation. For simplicity reasons, we write $\ell\in \mathcal{D}$, $x\in\mathcal{D}$ etc. (ii)
label $r$ is defined as final if there is an example in the hierarchy with label path of form $\ell/\dots/r$.

In the case that the  hierarchy is as general DAG as possible, we apply brute force algorithm. The open question is whether we can do better.

In the case that the hierarchy is also a tree, we compute (\ref{eqn:goal}) in linear time as follows:
\begin{itemize}
	\item we find two most distant label vectors in the hierarchy using the algorithm \ref{alg:rec} recursively:
	on each call, at most two candidates are returned
	\item we compute the distance between the two farthest candidates.
\end{itemize}

\begin{algorithm}[H]
	\caption{maxTree(root $r$)}\label{alg:rec}
	\begin{algorithmic}[1]
		\STATE{\texttt{candidates} $\gets$ empty list}
		\IF{$r\notin\mathcal{D}$}\label{alg:base}
			\RETURN{\texttt{candidates}}\hfill i.e., no example has label $r$
		\ENDIF
		\STATE{\texttt{chC} $\gets$ empty list}\hfill children candidates list
		\FORALL{$c\in$ $r$.children}
			\STATE{\texttt{cand} $\gets$ maxTree($c$)}
			\STATE{add \texttt{cand} to \texttt{chC} if \texttt{cand} not empty}\label{alg:emptySubtree}
		\ENDFOR
		\STATE{$n\gets$ length of \texttt{chC}}
		\IF{$n = 0$}
			\STATE{add $r$ to \texttt{candidates} if $r$ is final label}
		\ELSIF{$n = 1$}
			\STATE{add $\texttt{chC}[0][0]$ to \texttt{candidates}}
			\IF{$|\texttt{chC}[0]| = 1$}
				\STATE{add $r$ to \texttt{candidates} if $r$ is final label}
			\ELSE
				\STATE{\texttt{can1} $\gets \texttt{chC}[0]$}\label{alg:pair1}
				\STATE{\texttt{can2} $\gets$ $\{\texttt{chC}[0][0], r\}$  (if $r$ is final label)}\label{alg:pair2}
				\STATE{\texttt{candidates} $\gets$ the better of \texttt{can1} and \texttt{can2}}	
			\ENDIF
		\ELSE
			\STATE{\texttt{can1} $\gets$ the optimal pair from \texttt{chC} (if exists)}
			\STATE{$i_{1,2}\gets$ indices of the deepest and second deepest terms among $\texttt{chC}[i][0]$}
			\STATE{\texttt{can2} $\gets \{\texttt{chC}[i_1][0], \texttt{chC}[i_2][0]\}$}
			\STATE{\texttt{candidates} $\gets$ the better of \texttt{can1} and \texttt{can2}}
		\ENDIF
		\RETURN{\texttt{candidates}}
	\end{algorithmic}  
\end{algorithm}
Note that the first component of the non-empty list of $\leq 2$ candidates is the deepest
(among the final ones) in the current sub-hierarchy, since the maximising pair always contains
 a term that is the deepest (among the final ones).
 
If $r$ is not the final class (line \ref{alg:base}), the part of hierarchy rooted in in $r$ is not in the data, hence we
can stop here, otherwise we proceed.

Subtrees with no final classes (line \ref{alg:emptySubtree}) can be ignored. The others (their number is $n$) are processed further.

If $n = 0$, the root is the only candidate.

If $n = 1$, the child has either a pair of two best nodes or only one node.
If the former, we add $r$ ot the candidates, otherwise, we check which of the pairs in lines $\ref{alg:pair1}$ and $\ref{alg:pair2}$
yields greater distance. Both outcomes are possible.

If $n \geq 2$, then the root cannot be a part of the optimal pair. This is either one of the pairs from children or
the combination of two children pairs, where we took the first (= deepest) component of the two pairs.





\end{document}
