\section{Output}



\begin{itemize}
    \item \optionNameStyle{ShowModels}:
           \begin{itemize}
                \item \optionPossibleValues{}: a subset of \optionPossibleValuesList{Default, Original, Pruned, Others}
                \item \optionDefaultValue{}: \optionDefaultValueStyle{\{Default, Pruned, Others\}}
                \item \optionDescrption{}: specifies which models are shown in the \texttt{.out} file
           \end{itemize}
    \item \optionNameStyle{TrainErrors}:
           \begin{itemize}
                \item \optionPossibleValues{}: an element of \optionPossibleValuesList{Yes, No}
                \item \optionDefaultValue{}: \optionDefaultValueStyle{Yes}
                \item \optionDescrption{}: if set to \formatOneElement{Yes}, training errors will be computed and included in the \texttt{.out} file
           \end{itemize}
    \item \optionNameStyle{ValidErrors}:
           \begin{itemize}
                \item \optionPossibleValues{}: an element of \optionPossibleValuesList{Yes, No}
                \item \optionDefaultValue{}: \optionDefaultValueStyle{Yes}
                \item \optionDescrption{}: if set to \formatOneElement{Yes}, validation errors will be computed and included in the \texttt{.out} file ????????
           \end{itemize}
    \item \optionNameStyle{TestErrors}:
           \begin{itemize}
                \item \optionPossibleValues{}: an element of \optionPossibleValuesList{Yes, No}
                \item \optionDefaultValue{}: \optionDefaultValueStyle{Yes}
                \item \optionDescrption{}: if set to \formatOneElement{Yes}, testing errors will be computed and included in the \texttt{.out} file
           \end{itemize}
    \item \optionNameStyle{AllFoldModels}:
           \begin{itemize}
                \item \optionPossibleValues{}: an element of \optionPossibleValuesList{Yes, No}
                \item \optionDefaultValue{}: \optionDefaultValueStyle{Yes}
                \item \optionDescrption{}: if set to \formatOneElement{Yes}, \clus{} will output the model built in each fold of a cross-validation.
           \end{itemize}
    \item \optionNameStyle{AllFoldErrors}:
           \begin{itemize}
                \item \optionPossibleValues{}: an element of  \optionPossibleValuesList{Yes, No}
                \item \optionDefaultValue{}: \optionDefaultValueStyle{No}
                \item \optionDescrption{}:  if set to \formatOneElement{Yes}, \clus{} will output the test set error (and other evaluation measures) for each fold.
           \end{itemize}
    \item \optionNameStyle{AllFoldDatasets}:
           \begin{itemize}
                \item \optionPossibleValues{}: an element of \optionPossibleValuesList{Yes, No}
                \item \optionDefaultValue{}: \optionDefaultValueStyle{No}
                \item \optionDescrption{}: if set to \formatOneElement{Yes}, \clus{} will output the test set error (and other evaluation measures) for each fold.
           \end{itemize}
    \item \optionNameStyle{UnknownFrequency}:
           \begin{itemize}
                \item \optionPossibleValues{}: an element of  \optionPossibleValuesList{Yes, No}
                \item \optionDefaultValue{}: \optionDefaultValueStyle{No}
                \item \optionDescrption{}: if set to \formatOneElement{Yes}, \clus{} will show in each node of the tree the proportion of instances that had a missing value for the test in that node.
           \end{itemize}
    \item \optionNameStyle{BranchFrequency}:
           \begin{itemize}
                \item \optionPossibleValues{}: an element of \optionPossibleValuesList{Yes, No}
                \item \optionDefaultValue{}: \optionDefaultValueStyle{No}
                \item \optionDescrption{}:  if set to \formatOneElement{Yes}, \clus{} will show in each node of the tree, for each possible outcome of the test in that node, the proportion of instances that had that outcome.
           \end{itemize}
    \item \optionNameStyle{ShowInfo}:
           \begin{itemize}
                \item \optionPossibleValues{}: an element of \optionPossibleValuesList{Count, CountByTarget, Distribution, Index, NodePrototypes, Key}
                \item \optionDefaultValue{}: \optionDefaultValueStyle{\{Count\}}
                \item \optionDescrption{}: specifies the type of information shown in the models in \texttt{.out} file
           \end{itemize}
    \item \optionNameStyle{PrintModelAndExamples}:
           \begin{itemize}
                \item \optionPossibleValues{}: an element of  \optionPossibleValuesList{Yes, No}
                \item \optionDefaultValue{}: \optionDefaultValueStyle{No}
                \item \optionDescrption{}: if set to \formatOneElement{Yes}, the leaves of the models in the \texttt{.out} file will include some basic information about the distribution of values of the numeric
                descriptive attributes: minimal and maximal value, average and standard deviation.
           \end{itemize}
    \item \optionNameStyle{WriteErrorFile}: until this bug is fixed ...
           \begin{itemize}
                \item \optionPossibleValues{}: ???
                \item \optionDefaultValue{}: \optionDefaultValueStyle{No}
                \item \optionDescrption{}: ???
           \end{itemize}
    \item \optionNameStyle{WriteModelFile}:
           \begin{itemize}
                \item \optionPossibleValues{}: an element of \optionPossibleValuesList{Yes, No}
                \item \optionDefaultValue{}: \optionDefaultValueStyle{No}
                \item \optionDescrption{}: if set to \formatOneElement{Yes}, the \texttt{.model} file will be created
           \end{itemize}
    \item \optionNameStyle{WritePredictions}:
           \begin{itemize}
                \item \optionPossibleValues{}: a subset of \optionPossibleValuesList{Train,Test} or \formatOneElement{None}
                \item \optionDefaultValue{}: \optionDefaultValueStyle{\{None\}}
                \item \optionDescrption{}: If \formatOneElement{Train} is included, then the prediction for each training instance will be written to an ARFF output file {\tt {\em filename}.train.{\em i}.pred.arff} with $i$ the iteration. In a single run, $i = 1$. In a 10-fold cross-validation, $i$ will vary from 1 to 10. If \formatOneElement{Test}, then the predictions for each test instance will be written to the file
                {\tt {\em filename}.test.pred.arff}.
           \end{itemize}
    \item \optionNameStyle{GzipOutput}:
           \begin{itemize}
                \item \optionPossibleValues{}: an element of \optionPossibleValuesList{Yes, No}
                \item \optionDefaultValue{}: \optionDefaultValueStyle{No}
                \item \optionDescrption{}:  if set to \formatOneElement{Yes}, \clus{} will compress output in gzip file format (approx. 10 times smaller file sizes).
           \end{itemize}
    \doNotShowThis{\item \optionNameStyle{ModelIDFiles}:
           \begin{itemize}
                \item \optionPossibleValues{}: ???
                \item \optionDefaultValue{}: \optionDefaultValueStyle{No}
                \item \optionDescrption{}: ???
           \end{itemize}
    \item \optionNameStyle{WriteCurves}:
           \begin{itemize}
                \item \optionPossibleValues{}: ???
                \item \optionDefaultValue{}: \optionDefaultValueStyle{No}
                \item \optionDescrption{}: ???
           \end{itemize}
    \item \optionNameStyle{OutputPythonModel}:
           \begin{itemize}
                \item \optionPossibleValues{}: ???
                \item \optionDefaultValue{}: \optionDefaultValueStyle{No}
                \item \optionDescrption{}: ???
           \end{itemize}
    \item \optionNameStyle{OutputJSONModel}:
           \begin{itemize}
                \item \optionPossibleValues{}: ???
                \item \optionDefaultValue{}: \optionDefaultValueStyle{No}
                \item \optionDescrption{}: ???
           \end{itemize}
    \item \optionNameStyle{OutputDatabaseQueries}:
           \begin{itemize}
                \item \optionPossibleValues{}: ???
                \item \optionDefaultValue{}: \optionDefaultValueStyle{No}
                \item \optionDescrption{}: ???
           \end{itemize}
    \item \optionNameStyle{OutputClowdFlowsJSON}:
           \begin{itemize}
                \item \optionPossibleValues{}: ???
                \item \optionDefaultValue{}: \optionDefaultValueStyle{No}
                \item \optionDescrption{}: ???
           \end{itemize}
      }
\end{itemize}


