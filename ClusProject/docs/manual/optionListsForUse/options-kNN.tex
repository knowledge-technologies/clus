\section{kNN}


\begin{itemize}
    \item \optionNameStyle{K}:
    \begin{itemize}
        \item \optionPossibleValues{}: a positive integer or a list thereof, e.g., \texttt{10} or \texttt{[1,5,10]}.
        \item \optionDefaultValue{}: \optionDefaultValueStyle{[1,3]}
        \item \optionDescrption{}: the number(s) of neighbors in the kNN algorithm. Computational complexity of the algorithm equals
        the complexity for the maximal number of neighbors.
    \end{itemize}
    \item \optionNameStyle{Distance}:
    \begin{itemize}
        \item \optionPossibleValues{}: \texttt{euclidean}, \texttt{manhattan} or \texttt{chebyshev}
        \item \optionDefaultValue{}: \optionDefaultValueStyle{euclidean}
        \item \optionDescrption{}: the distance measure which is used when finding the nearest neighbours.        
        This option (together with the option \optionNameStyle{AttributeWeighting}) defines the distance measures used in nearest neighbour search:
        \begin{itemize}
            \item \optionDefaultValueStyle{euclidean}: $d(a, b) = \sqrt{\sum_i w_i d_i(a_i, b_i)^2}$,
            \item \optionDefaultValueStyle{manhattan}: $d(a, b) = \sum_i w_i d_i(a_i, b_i)$,
            \item \optionDefaultValueStyle{chebyshev}: $d(a, b) = \max_i w_i d_i(a_i, b_i)$,
        \end{itemize}
        where $a$ and $b$ are vectors of descriptive attributes, the weights $w_i$ are defined by the option \optionNameStyle{AttributeWeighting},
        and $d_i$ is the distance in the space of the $i$-th attribute.
        
        If the $i$-th attribute is numeric, then $d_i(a_i, b_i)$ is proportional to $|a_i - b_i|$ and linearly normalized to the interval $[0, 1]$.
        If the $i$-th attribute is nominal, then $d_i(a_i, b_i) = 0$ if $a_i = b_i$ and  $d_i(a_i, b_i) = 1$ otherwise.
    \end{itemize}
    \item \optionNameStyle{SearchMethod}:
    \begin{itemize}
        \item \optionPossibleValues{}: {\tt BruteForce}, {\tt VP-tree} or {\tt KD-tree}
        \item \optionDefaultValue{}: \optionDefaultValueStyle{BrutForce}
        \item \optionDescrption{}: the method that is used when finding the nearest neighbors. If the number of attributes is high, {\tt BruteForce}
        may be the fastest, since the structures such as VP- and KD-trees suffer from the \emph{curse of dimensionality}.
    \end{itemize}
    \item \optionNameStyle{DistanceWeighting}:
    \begin{itemize}
        \item \optionPossibleValues{}: {\tt constant}, {\tt 1-d} or {\tt 1/d}
        \item \optionDefaultValue{}: \optionDefaultValueStyle{constant}
        \item \optionDescrption{}: When making predictions for a new example, this option is used to weight the contributions of the neighbors. The option {\tt constant}
        is equivalent to no weighting, whereas the options {\tt 1-d} and {\tt 1/d} respectively correspond to the weights that equal
        $1 - d(\text{example, neighbour})$ and $1 / (\varepsilon + d(\text{example, neighbour}))$, where $\varepsilon = 0.001$ prevents zero-division
        problems.
    \end{itemize}
    \item \optionNameStyle{AttributeWeighting}:
    \begin{itemize}
        \item \optionPossibleValues{}: a string that equals {\tt none}, or starts with {\tt RF}, or represents a list of double values, e.g.,
        {\tt none} or {\tt RF} or {\tt RF,100} or {\tt [2.71828,3.14,1.618,0.0]}.
        \item \optionDefaultValue{}: \optionDefaultValueStyle{none}
        \item \optionDescrption{}: From the value of these option, the dimensional weights $w_i$ are computed, i.e., the weights that control the contribution
        of each dimension (attribute). The weights are defined as follows:
        \begin{itemize}
            \item {\tt none}: $w_i = 1$
            \item {\tt RF,<number of trees>}: $w_i$ equals the importance of the $i$-th attribute as computed from {\tt RForest} ranking from the ensemble
            of the size {\tt <number of trees>}.
            \item {\tt RF}: equivalent to {\tt RF,100}
            \item {\tt <explicitly given list of weights>}: $w_i$ equals the $i$-th element of the list.
        \end{itemize}
    \end{itemize}
\end{itemize}